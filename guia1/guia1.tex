\documentclass[10pt]{article}
%\usepackage{wallpaper}
\usepackage[spanish]{babel}
\usepackage[utf8]{inputenc}
\usepackage[top=1in, bottom=1.0in, left=1.5in, right=1.5in]{geometry}

\usepackage{amsmath}
\usepackage{graphicx}
\usepackage{animate}
\usepackage{amssymb}
\usepackage{gensymb}
\usepackage[makeroom]{cancel}
\usepackage{mathtools}% Loads amsmath



\pagestyle{empty}
%\setlength\parindent{0pt}

\begin{document}

%\vspace{1.5cm}
\hfill{Buenos Aires, \today} \\

\begin{center}
 {\large \bf Estructura Electrónica de Materias: \\
 Cálculo desde primeros principios} \\
 
 \vspace{0.25cm}
 Guía Práctica N\degree 1
\end{center}

\vspace{0.5cm}
\noindent
1. {\bf Principio Variacional}:

\vspace{0.25cm}
Siendo $\left|\psi_n\right>$ la solución exacta de la ecuación de Schr\"odinger
independiente del tiempo de un sistema de N partículas,
\begin{equation}
 \hat{H}\left|\psi_n\right> = E_n\left|\psi_n\right>\,
\end{equation}
y $\left|\epsilon\right>$ un vector que representa un error pequeño.
Si el autovalor solución $\left|\phi\right>$ que se obtiene del 
principio variacional difiere de la solución exacta por $\left|\epsilon\right>$: 
\begin{equation}
 \phi=\left|\psi_n\right>+\left|\epsilon\right>\,,
\end{equation}
entonces, el error en la energía, $E[\phi]-E_n$, es de segundo orden.

\begin{center}
 \noindent\rule[0.5ex]{0.9\linewidth}{0.1pt}
\end{center}

Del principio variacional, se define el funcional de la energía
\begin{equation}
 E[\phi] = \frac{\left< \phi \right| \hat{H} \left| \phi \right>}{\left< \phi | \phi \right>} \,.
 \label{eq:prinvar}
\end{equation}
Si la autofunción solución puede escribirse como
\begin{equation}
 \left|\phi\right>=\left|\psi_n\right>+\left|\epsilon\right>\,
\end{equation}
y si $\left<\psi_n|\psi_n\right>=\left<\phi|\phi\right>=1$, entonces, la ecuación
(\ref{eq:prinvar}) puede escribirse como
\begin{eqnarray}
 E\left[\phi\right] 
 &=& \left< \psi_n+\epsilon \right| \hat{H} \left| \psi_n+\epsilon \right>  \\
 &=& \left< \psi_n \right| \hat{H} \left| \psi_n \right> 
     + \left< \psi_n \right| \hat{H} \left| \epsilon \right> 
     + \left< \epsilon \right| \hat{H} \left| \psi_n \right> 
     + \left< \epsilon \right| \hat{H} \left| \epsilon \right> \\
 &=& E_n \underbrace{\left< \psi_n | \psi_n \right>}_{=1}
  + \cancel{\left< \psi_n \right| \hat{H} \left| \epsilon \right>} 
  + \cancel{\left< \epsilon \right| \hat{H} \left| \psi_n \right>} 
  + \left< \epsilon \right| \hat{H} \left| \epsilon \right> \\
 &=& E_n + \left< \epsilon \right| \hat{H} \left| \epsilon \right> \\
 &\Rightarrow& \quad
 E_n - E\left[\phi\right] = \mathcal{O}^2(\epsilon)
\end{eqnarray}

\vspace{0.5cm}
\noindent
2. {\bf Método de Hartree--Fock}:

\vspace{0.25cm}
El Hamiltoniano de dos electrones se escribe como:
\begin{equation}
 \hat{H} = -\frac{1}{2}\sum_{i=1}^2 \nabla^2 + \sum_{i=1}^2 v(\mathbf{r}_i) 
 + \sum_{i<j}^2 \frac{1}{\mathbf{r}_{ij}}\,.
\end{equation}
Asumiendo que la función de onda del sistema está dada por
\begin{equation}
 \Psi^{\mathrm{HF}}(\mathbf{q}_1,\mathbf{q}_2)=\frac{1}{2}\left[
 \psi_n(\mathbf{q}_1)\psi_m(\mathbf{q}_2)-\psi_n(\mathbf{q}_2)\psi_m(\mathbf{q}_1)
 \right]\,,
\end{equation}
donde $q_i$ representa las coordenadas espaciales y de espín, 
la energía total de Hartree Fock del sistema resulta:
La energía de Hartree--Fock está dada por:
\begin{align}
 E^{\mathrm{HF}}
 &=\left<\Psi^{\mathrm{HF}}\right|\hat{H}\left|\Psi^{\mathrm{HF}}\right> \\
 &=\left<\Psi^{\mathrm{HF}}\right| -\frac{1}{2}\sum_{i=1}^2 \nabla^2 + \sum_{i=1}^2 v(\mathbf{r}_i) 
 + \sum_{i<j}^2 \frac{1}{\mathbf{r}_{ij}} \left|\Psi^{\mathrm{HF}}\right> \\
 &=\sum_{n=1}^2 \left[ I_n 
 + \frac{1}{2} \sum_{m=1}^{2}\left(J_{nm}-K_{nm}\right) \right] 
\end{align}
donde 
\begin{equation}
 I_n=\int\psi_n^*(\mathbf{q})\left[-\frac{1}{2}\nabla^2-v(\mathbf{r})\right]\psi_n(\mathbf{q}) \,d\mathbf{q}
\end{equation}
\begin{equation}
 J_{nm}=\iint 
\end{equation}




\end{document}
