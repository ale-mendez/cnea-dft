\documentclass[10pt]{article}
%\usepackage{wallpaper}
\usepackage[spanish]{babel}
\usepackage[utf8]{inputenc}
\usepackage[top=1in, bottom=1.0in, left=1.in, right=1.in]{geometry}

\usepackage{amsmath}
\usepackage{graphicx}
\usepackage{animate}
\usepackage{amssymb}
\usepackage{gensymb}
\usepackage[makeroom]{cancel}
\usepackage{mathtools}% Loads amsmath
\usepackage{tikz}

\newlength\myheight
\newcommand*\ccircled[1]{\settowidth{\myheight}{#1}%
    \raisebox{-.1\myheight}{\tikz[baseline=(char.base)]{%
        \node[shape=circle,draw,minimum size=1.5em,inner sep=1pt](char){#1};}}}
        
        
\pagestyle{empty}
\setlength\parindent{0pt}

\begin{document}

%\vspace{1.5cm}
%\hfill{Buenos Aires, \today} \\

\begin{center}
 {\large \bf Estructura Electrónica de Materias: \\
 Cálculo desde primeros principios} \\
 
 \vspace{0.25cm}
 Guía Práctica N\degree 1
\end{center}

\vspace{0.5cm}
\noindent
1. {\bf Principio Variacional}:

\vspace{0.25cm}
Siendo $\left|\psi_n\right>$ la solución exacta de la ecuación de Schr\"odinger
independiente del tiempo de un sistema de N partículas,
\begin{equation}
 \hat{H}\left|\psi_n\right> = E_n\left|\psi_n\right>\,
\end{equation}
y $\left|\epsilon\right>$ un vector que representa un error pequeño.
Si el autovalor solución $\left|\phi\right>$ que se obtiene del 
principio variacional difiere de la solución exacta por $\left|\epsilon\right>$: 
\begin{equation}
 \phi=\left|\psi_n\right>+\left|\epsilon\right>\,,
\end{equation}
entonces, el error en la energía, $E[\phi]-E_n$, es de segundo orden.

\begin{center}
 \noindent\rule[0.5ex]{0.9\linewidth}{0.1pt}
\end{center}

Del principio variacional, se define el funcional de la energía
\begin{equation}
 E[\phi] = \frac{\left< \phi \right| \hat{H} \left| \phi \right>}{\left< \phi | \phi \right>} \,.
 \label{eq:prinvar}
\end{equation}
Si la autofunción solución puede escribirse como
\begin{equation}
 \left|\phi\right>=\left|\psi_n\right>+\left|\epsilon\right>\,
\end{equation}
y si $\left<\psi_n|\psi_n\right>=\left<\phi|\phi\right>=1$, entonces, la ecuación
(\ref{eq:prinvar}) puede escribirse como
\begin{eqnarray}
 E\left[\phi\right] 
 &=& \left< \psi_n+\epsilon \right| \hat{H} \left| \psi_n+\epsilon \right>  \\
 &=& \left< \psi_n \right| \hat{H} \left| \psi_n \right> 
     + \left< \psi_n \right| \hat{H} \left| \epsilon \right> 
     + \left< \epsilon \right| \hat{H} \left| \psi_n \right> 
     + \left< \epsilon \right| \hat{H} \left| \epsilon \right> \\
 &=& E_n \underbrace{\left< \psi_n | \psi_n \right>}_{=1}
  + \cancel{\left< \psi_n \right| \hat{H} \left| \epsilon \right>} 
  + \cancel{\left< \epsilon \right| \hat{H} \left| \psi_n \right>} 
  + \left< \epsilon \right| \hat{H} \left| \epsilon \right> \\
 &=& E_n + \left< \epsilon \right| \hat{H} \left| \epsilon \right> \\
 &\Rightarrow& \quad
 E_n - E\left[\phi\right] = \mathcal{O}^2(\epsilon)
\end{eqnarray}

\vspace{0.5cm}
\noindent
2. {\bf Método de Hartree--Fock}:

\vspace{0.25cm}
El Hamiltoniano de dos electrones se escribe como:
\begin{align}
 \hat{H} &= -\frac{1}{2}\sum_{i=1}^2 \nabla_{\mathrm{r}_i}^2 
 + \sum_{i=1}^2 v(\mathbf{r}_i) + \sum_{i<j}^2 \frac{1}{r_{ij}} \\
 &= -\frac{1}{2} \nabla_{\mathrm{r}_1}^2 -\frac{1}{2} \nabla_{\mathrm{r}_1}^2 
 +  v(\mathbf{r}_1) + v(\mathbf{r}_2) + \frac{1}{r_{12}} \\
 &= \left[-\frac{1}{2} \nabla_{\mathrm{r}_1}^2 +  v(\mathbf{r}_1) \right] +
 \left[-\frac{1}{2} \nabla_{\mathrm{r}_1}^2 + v(\mathbf{r}_2) \right]
 + \frac{1}{r_{12}} \\
 &= \hat{h}_1 + \hat{h}_2 + \frac{1}{r_{12}}
\end{align}
Asumiendo que la función de onda del sistema está dada por
\begin{equation}
 \Psi^{\mathrm{HF}}(\mathbf{q}_1,\mathbf{q}_2)=\frac{1}{\sqrt{2}}\left[
 \psi_n(\mathbf{q}_1)\psi_m(\mathbf{q}_2)-\psi_n(\mathbf{q}_2)\psi_m(\mathbf{q}_1)
 \right]\,,
 \label{eq:slater2e}
\end{equation}
donde $\mathbf{q}_i$ representa las coordenadas espaciales y de espín, 
la energía total de Hartree Fock del sistema resulta:
\begin{align}
 E^{\mathrm{HF}}
 &=\left<\Psi^{\mathrm{HF}}\right|\hat{H}\left|\Psi^{\mathrm{HF}}\right> \\
 %%%%
 &=\frac{1}{2}
 \left< \psi_n(\mathbf{q}_1)\psi_m(\mathbf{q}_2)
 -\psi_n(\mathbf{q}_2)\psi_m(\mathbf{q}_1) \right| \hat{H} 
 \left| \psi_n(\mathbf{q}_1)\psi_m(\mathbf{q}_2)
 -\psi_n(\mathbf{q}_2)\psi_m(\mathbf{q}_1)  \right> \\
 %%%%
 &=\frac{1}{2} \big[ 
 \underbrace{\left< \psi_n(\mathbf{q}_1)\psi_m(\mathbf{q}_2)  \right| \hat{H}
 \left| \psi_n(\mathbf{q}_1)\psi_m(\mathbf{q}_2) \right>}_{\ccircled{A}} -
 %%
 \underbrace{\left< \psi_n(\mathbf{q}_1)\psi_m(\mathbf{q}_2)  \right| \hat{H}
 \left| \psi_n(\mathbf{q}_2)\psi_m(\mathbf{q}_1) \right>}_{\ccircled{B}} \\
 &\quad
 %%
 - \underbrace{\left< \psi_n(\mathbf{q}_2)\psi_m(\mathbf{q}_1) \right| \hat{H}
 \left| \psi_n(\mathbf{q}_1)\psi_m(\mathbf{q}_2) \right>}_{\ccircled{C}} + 
 %%
 \underbrace{\left< \psi_n(\mathbf{q}_2)\psi_m(\mathbf{q}_1) \right| \hat{H}
 \left| \psi_n(\mathbf{q}_2)\psi_m(\mathbf{q}_1) \right> }_{\ccircled{D}}
 \big] 
 \end{align}
\begin{align}
 \ccircled{A}
 &=\iint \psi_n^*(\mathbf{q}_1)\psi_m^*(\mathbf{q}_2) 
 \left[ \hat{h}_1 + \hat{h}_2 + \frac{1}{r_{12}} \right] 
 \psi_n(\mathbf{q}_1)\psi_m(\mathbf{q}_2) \,d\mathbf{q}_1 d\mathbf{q}_2 \\
 %%%%%
%  &=\iint \psi_n^*(\mathbf{q}_1)\psi_m^*(\mathbf{q}_2) \hat{h}_1 
%   \psi_n(\mathbf{q}_1)\psi_m(\mathbf{q}_2) \,d\mathbf{q}_1 d\mathbf{q}_2 + \\
%  &\qquad\quad
%  \iint \psi_n^*(\mathbf{q}_1)\psi_m^*(\mathbf{q}_2) \hat{h}_2 
%  \psi_n(\mathbf{q}_1)\psi_m(\mathbf{q}_2) \,d\mathbf{q}_1 d\mathbf{q}_2 + \\
%  &\qquad\qquad\quad
%  \iint \psi_n^*(\mathbf{q}_1)\psi_m^*(\mathbf{q}_2) 
%  \frac{1}{r_{12}} 
%  \psi_n(\mathbf{q}_1)\psi_m(\mathbf{q}_2) \,d\mathbf{q}_1 d\mathbf{q}_2 \\
 %%%%%
 &=\int \psi_n^*(\mathbf{q}_1)\, \hat{h}_1 \, \psi_n(\mathbf{q}_1) \,d\mathbf{q}_1
  \underbrace{\int \psi_m^*(\mathbf{q}_2) \psi_m(\mathbf{q}_2) \, d\mathbf{q}_2}_{=1} \\
 &\qquad\quad +
 \underbrace{\int \psi_n^*(\mathbf{q}_1)\psi_n(\mathbf{q}_1)\,d\mathbf{q}_1}_{=1}  
 \int \psi_m^*(\mathbf{q}_2) \, \hat{h}_2 \, \psi_m(\mathbf{q}_2) \,d\mathbf{q}_2 \\
 &\qquad\qquad\quad +
 \int \psi_n^*(\mathbf{q}_1)\psi_m^*(\mathbf{q}_2) 
 \frac{1}{r_{12}} 
 \psi_n(\mathbf{q}_1)\psi_m(\mathbf{q}_2) \,d\mathbf{q}_1 d\mathbf{q}_2 \\
 %%%%%
 &=\int \psi_n^*(\mathbf{q}_1)\, \hat{h}_1 \, \psi_n(\mathbf{q}_1) \,d\mathbf{q}_1 + 
 \int \psi_m^*(\mathbf{q}_2) \, \hat{h}_2 \, \psi_m(\mathbf{q}_2) \,d\mathbf{q}_2 + \\
 &\qquad\qquad\quad
 \int \psi_n^*(\mathbf{q}_1)\psi_m^*(\mathbf{q}_2) 
 \frac{1}{r_{12}} 
 \psi_n(\mathbf{q}_1)\psi_m(\mathbf{q}_2) \,d\mathbf{q}_1 d\mathbf{q}_2 
\end{align}

\begin{align}
 \ccircled{B}
 &=\iint \psi_n^*(\mathbf{q}_1)\psi_m^*(\mathbf{q}_2)  
 \left[ \hat{h}_1 + \hat{h}_2 + \frac{1}{r_{12}} \right] 
 \psi_n(\mathbf{q}_2)\psi_m(\mathbf{q}_1) \,d\mathbf{q}_1 d\mathbf{q}_2 \\
%  &=\iint \psi_n^*(\mathbf{q}_1)\psi_m^*(\mathbf{q}_2)\,\hat{h}_1\,
%          \psi_n(\mathbf{q}_2)\psi_m(\mathbf{q}_1) \,d\mathbf{q}_1 d\mathbf{q}_2
%  +\\ &\qquad\quad
%    \iint \psi_n^*(\mathbf{q}_1)\psi_m^*(\mathbf{q}_2)\, \hat{h}_2\,
%          \psi_n(\mathbf{q}_2)\psi_m(\mathbf{q}_1) \,d\mathbf{q}_1 d\mathbf{q}_2
%   + \\ &\qquad\qquad\quad
%    \iint \psi_n^*(\mathbf{q}_1)\psi_m^*(\mathbf{q}_2) \frac{1}{r_{12}} 
%          \psi_n(\mathbf{q}_2)\psi_m(\mathbf{q}_1)  \,d\mathbf{q}_1 d\mathbf{q}_2 \\
 &=\int \psi_n^*(\mathbf{q}_1)\,\hat{h}_1\,\psi_m(\mathbf{q}_1) \,d\mathbf{q}_1
   \cancel{\int \psi_m^*(\mathbf{q}_2) \psi_n(\mathbf{q}_2) d\mathbf{q}_2}
 +\\ &\qquad\quad
  \cancel{ \int \psi_n^*(\mathbf{q}_1)\psi_m(\mathbf{q}_1) \,d\mathbf{q}_1}
   \int \psi_m^*(\mathbf{q}_2)\,\hat{h}_2\,\psi_n(\mathbf{q}_2) d\mathbf{q}_2
  + \\ &\qquad\qquad\quad
   \iint \psi_n^*(\mathbf{q}_1)\psi_m^*(\mathbf{q}_2) \frac{1}{r_{12}} 
         \psi_n(\mathbf{q}_2)\psi_m(\mathbf{q}_1)  \,d\mathbf{q}_1 d\mathbf{q}_2 \\
%%%%
 &=\iint \psi_n^*(\mathbf{q}_1)\psi_m^*(\mathbf{q}_2) \frac{1}{r_{12}} 
         \psi_n(\mathbf{q}_2)\psi_m(\mathbf{q}_1)  \,d\mathbf{q}_1 d\mathbf{q}_2 
\end{align}

\begin{align}
 \ccircled{C}
 &=\iint \psi_n^*(\mathbf{q}_2)\psi_m^*(\mathbf{q}_1)
 \left[ \hat{h}_1 + \hat{h}_2 + \frac{1}{r_{12}} \right] 
 \psi_n(\mathbf{q}_1)\psi_m(\mathbf{q}_2) \,d\mathbf{q}_1 d\mathbf{q}_2 \\
%  &=\iint \psi_n^*(\mathbf{q}_2)\psi_m^*(\mathbf{q}_1)\,\hat{h}_1\,
%          \psi_n(\mathbf{q}_1)\psi_m(\mathbf{q}_2)\,d\mathbf{q}_1 d\mathbf{q}_2
%  +\\ &\qquad\quad
%    \iint \psi_n^*(\mathbf{q}_2)\psi_m^*(\mathbf{q}_1)\,\hat{h}_2\,
%          \psi_n(\mathbf{q}_1)\psi_m(\mathbf{q}_2)\,d\mathbf{q}_1 d\mathbf{q}_2
%   + \\ &\qquad\qquad\quad
%    \iint \psi_n^*(\mathbf{q}_2)\psi_m^*(\mathbf{q}_1) \frac{1}{r_{12}} 
%          \psi_n(\mathbf{q}_1)\psi_m(\mathbf{q}_2) \,d\mathbf{q}_1 d\mathbf{q}_2 \\
%%%%%
 &=\int \psi_m^*(\mathbf{q}_1)\,\hat{h}_1\, \psi_n(\mathbf{q}_1)\,d\mathbf{q}_1
   \cancel{\int \psi_n^*(\mathbf{q}_2)\psi_m(\mathbf{q}_2) \,d\mathbf{q}_2}
 +\\ &\qquad\quad
   \cancel{\int \psi_m^*(\mathbf{q}_1)\psi_n(\mathbf{q}_1)\,d\mathbf{q}_1}
   \int \psi_n^*(\mathbf{q}_2)\,\hat{h}_2\,\psi_m(\mathbf{q}_2)\,d\mathbf{q}_2
  + \\ &\qquad\qquad\quad
   \iint \psi_n^*(\mathbf{q}_2)\psi_m^*(\mathbf{q}_1) \frac{1}{r_{12}} 
         \psi_n(\mathbf{q}_1)\psi_m(\mathbf{q}_2) \,d\mathbf{q}_1 d\mathbf{q}_2 \\
 &= \iint \psi_n^*(\mathbf{q}_2)\psi_m^*(\mathbf{q}_1) \frac{1}{r_{12}} 
         \psi_n(\mathbf{q}_1)\psi_m(\mathbf{q}_2) \,d\mathbf{q}_1 d\mathbf{q}_2 
\end{align}

\begin{align}
 \ccircled{D}
 &= \iint \psi_n^*(\mathbf{q}_2)\psi_m^*(\mathbf{q}_1) 
 \left[ \hat{h}_1 + \hat{h}_2 + \frac{1}{r_{12}} \right] 
 \psi_n(\mathbf{q}_2)\psi_m(\mathbf{q}_1) \,d\mathbf{q}_1 d\mathbf{q}_2 \\
%  &= \iint \psi_n^*(\mathbf{q}_2)\psi_m^*(\mathbf{q}_1)\,\hat{h}_1\,
%  \psi_n(\mathbf{q}_2)\psi_m(\mathbf{q}_1) \,d\mathbf{q}_1 d\mathbf{q}_2 
%  +\\ &\qquad\quad
%  \iint \psi_n^*(\mathbf{q}_2)\psi_m^*(\mathbf{q}_1)\,\hat{h}_2\,
%  \psi_n(\mathbf{q}_2)\psi_m(\mathbf{q}_1)\,d\mathbf{q}_1 d\mathbf{q}_2 
%   + \\ &\qquad\qquad\quad
%  \iint \psi_n^*(\mathbf{q}_2)\psi_m^*(\mathbf{q}_1)\frac{1}{r_{12}}
%  \psi_n(\mathbf{q}_2)\psi_m(\mathbf{q}_1)\,d\mathbf{q}_1 d\mathbf{q}_2  \\
 %%%%%
 &= \int \psi_m^*(\mathbf{q}_1)\,\hat{h}_1\,\psi_m(\mathbf{q}_1) \,d\mathbf{q}_1
 \underbrace{\int\psi_n^*(\mathbf{q}_2)\psi_n(\mathbf{q}_2)\,\,d\mathbf{q}_2}_{=1}
 +\\ &\qquad\quad
 \underbrace{ \int \psi_m^*(\mathbf{q}_1)\psi_m(\mathbf{q}_1)\,d\mathbf{q}_1}_{=1}
 \int \psi_n^*(\mathbf{q}_2)\,\hat{h}_2\,\psi_n(\mathbf{q}_2)\,d\mathbf{q}_2
  + \\ &\qquad\qquad\quad
 \iint \psi_n^*(\mathbf{q}_2)\psi_m^*(\mathbf{q}_1)\frac{1}{r_{12}}
 \psi_n(\mathbf{q}_2)\psi_m(\mathbf{q}_1) \\
 &=\int \psi_m^*(\mathbf{q}_1)\,\hat{h}_1\,\psi_m(\mathbf{q}_1) \,d\mathbf{q}_1+
 \int \psi_n^*(\mathbf{q}_2)\,\hat{h}_2\,\psi_n(\mathbf{q}_2)\,d\mathbf{q}_2
  + \\ &\qquad\qquad\quad
 \iint \psi_n^*(\mathbf{q}_2)\psi_m^*(\mathbf{q}_1)\frac{1}{r_{12}}
 \psi_n(\mathbf{q}_2)\psi_m(\mathbf{q}_1) 
\end{align}

\begin{align}
 E^{\mathrm{HF}}
 &=\frac{1}{2}\left[ \sum_{i=1}^2 \left< \psi_i(\mathbf{q}_1)\right|\hat{h}_1\left| \psi_i(\mathbf{q}_1)\right> + 
 \sum_{i=1}^2 \left<\psi_i(\mathbf{q}_2)\right|\hat{h}_2\left|\psi_i(\mathbf{q}_2)\right> \right]
 \\ &\qquad
+\frac{1}{2}\bigg[\left<\psi_n(\mathbf{q}_1)\psi_m(\mathbf{q}_2) \right|
\frac{1}{r_{12}} \left|\psi_n(\mathbf{q}_1)\psi_m(\mathbf{q}_2) \right>
-\left<\psi_n(\mathbf{q}_1)\psi_m(\mathbf{q}_2)\right|\frac{1}{r_{12}}\left|
\psi_m(\mathbf{q}_1)\psi_n(\mathbf{q}_2)\right>
 \\ &\qquad\qquad\quad
-\left<\psi_m(\mathbf{q}_1)\psi_n(\mathbf{q}_2)\right|\frac{1}{r_{12}}\left|
\psi_n(\mathbf{q}_1)\psi_m(\mathbf{q}_2)\right>
+\left<\psi_m(\mathbf{q}_1)\psi_n(\mathbf{q}_2)\right|\frac{1}{r_{12}}\left|
 \psi_m(\mathbf{q}_1)\psi_n(\mathbf{q}_2) \right> \bigg]\\
%%%%%
 &=\sum_{i=1}^2 \left< \psi_i(\mathbf{q})\right|\hat{h}\left| \psi_i(\mathbf{q})\right> 
 \\ &\qquad
+\frac{1}{2}\bigg[
\left<\psi_n(\mathbf{q}_1)\psi_m(\mathbf{q}_2)
\right|\frac{1}{r_{12}} \left|
\psi_n(\mathbf{q}_1)\psi_m(\mathbf{q}_2) \right> +
\left<\psi_n(\mathbf{q}_2)\psi_m(\mathbf{q}_1)
\right|\frac{1}{r_{12}} \left|
 \psi_n(\mathbf{q}_2)\psi_m(\mathbf{q}_1)\right>
 \\ &\qquad\qquad\quad
-\left<\psi_n(\mathbf{q}_1)\psi_m(\mathbf{q}_2)\right|\frac{1}{r_{12}}\left|
\psi_m(\mathbf{q}_1)\psi_n(\mathbf{q}_2)\right>
-\left<\psi_m(\mathbf{q}_1)\psi_n(\mathbf{q}_2)\right|\frac{1}{r_{12}}\left|
\psi_n(\mathbf{q}_1)\psi_m(\mathbf{q}_2)\right>\bigg] 
\end{align}
Dado que los electrones son indistinguibles, tenemos
\begin{align}
 E^{\mathrm{HF}}
%%%%%%
 &=\sum_{i=1}^2 \left< \psi_i(\mathbf{q})\right|\hat{h}\left| \psi_i(\mathbf{q})\right> 
 \\ &\qquad
+\frac{1}{2}\bigg[
\left<\psi_n(\mathbf{q}_1)\psi_m(\mathbf{q}_2)
\right|\frac{1}{r_{12}} \left|
\psi_n(\mathbf{q}_1)\psi_m(\mathbf{q}_2) \right> 
+\left<\psi_n(\mathbf{q}_1)\psi_m(\mathbf{q}_2)
\right|\frac{1}{r_{21}} \left|
 \psi_n(\mathbf{q}_1)\psi_m(\mathbf{q}_2)\right>
 \\ &\qquad\qquad\quad
-\left<\psi_n(\mathbf{q}_1)\psi_m(\mathbf{q}_2)\right|\frac{1}{r_{12}}\left|
\psi_m(\mathbf{q}_1)\psi_n(\mathbf{q}_2)\right>
-\left<\psi_m(\mathbf{q}_2)\psi_n(\mathbf{q}_1)\right|\frac{1}{r_{21}}\left|
\psi_n(\mathbf{q}_2)\psi_m(\mathbf{q}_1)\right>\bigg] \\
%%%%%%
 &=\underbrace{\sum_{i=1}^2 \left< \psi_i(\mathbf{q})\right|
 -\frac{1}{2}\nabla^2 \left| \psi_i(\mathbf{q})\right>}_{
 \mathrm{Energia\,cinetica\,de\,particula\,indep.}}
 +\underbrace{\sum_{i=1}^2 \left< \psi_i(\mathbf{q})\right| 
 v(\mathbf{r}_i) \left| \psi_i(\mathbf{q})\right>}_{\mathrm{Energia\,debido\,a\,potencial\,externo}} \\
 &\qquad\qquad
+\underbrace{\left<\psi_n(\mathbf{q}_1)\psi_m(\mathbf{q}_2)
\right|\frac{1}{r_{12}} \left|
\psi_n(\mathbf{q}_1)\psi_m(\mathbf{q}_2) \right> }_{\mathrm{
Termino\,directo}}
-\underbrace{\left<\psi_n(\mathbf{q}_1)\psi_m(\mathbf{q}_2)\right|\frac{1}{r_{12}}\left|
\psi_m(\mathbf{q}_1)\psi_n(\mathbf{q}_2)\right>}_{\mathrm{Termino\,de\,intercambio}}
\end{align}
En las ecuaciones de Hartree--Fock, el término de la energía directa está dado por la repulsión Coulombiana entre el electrón 1
en el orbital $n$ y el electron 2 en el orbital $m$. Por otro lado,
el término de la energía de intercambio no tiene una explicación
física, sino que surge de la condición de indistinguibilidad dada
por el determinante de Slater definido en la ecuación 
(\ref{eq:slater2e}). 

\vspace{0.5cm}
Los términos directo y de intercambio en un sistema de un electrón,
en donde $n=m$, resultan
\begin{align}
J_{n,n}&= \left<\psi_n(\mathbf{q}_1)\psi_n(\mathbf{q}_2)
\right|\frac{1}{r_{12}} \left|
\psi_n(\mathbf{q}_1)\psi_n(\mathbf{q}_2) \right> \\
K_{n,m}&=\left<\psi_n(\mathbf{q}_1)\psi_n(\mathbf{q}_2)\right|\frac{1}{r_{12}}\left|
\psi_n(\mathbf{q}_1)\psi_n(\mathbf{q}_2)\right> \\
&\qquad \Rightarrow \quad J_{n,n}=K_{n,n}
\end{align}

\vspace{0.5cm}
La función de onda de un electrón se puede descomponer en sus
coordenadas espacial y de espín como
\begin{equation}
 \psi_n(\mathbf{q}) = \phi_n(\mathbf{r})\chi_n(\omega)\,.
\end{equation}
El término de la energía de interacción resulta
\begin{align}
 K_{n,m}
 &=\int 
 \frac{\phi_n^*(\mathbf{r}_1)\chi_n^*(\omega_1)\phi_m^*(\mathbf{r}_2)\chi_m^*(\omega_2)\phi_m(\mathbf{r}_1)\chi_m(\omega_1)\phi_n(\mathbf{r}_2)\chi_n(\omega_2)}{r_{12}} \,d\mathbf{r}_1\,d\mathbf{r}_2 \,d\omega_1\,d\omega_2\\
 &=\int 
 \frac{\phi_n^*(\mathbf{r}_1)\phi_m^*(\mathbf{r}_2)\phi_m(\mathbf{r}_1)\phi_n(\mathbf{r}_2)}{r_{12}} 
 \,d\mathbf{r}_1\,d\mathbf{r}_2 
 \int \chi_n^*(\omega_1)\chi_m(\omega_1)\,d\omega_1
 \int \chi_m^*(\omega_2)\chi_n(\omega_2) \,d\omega_2\\
\end{align}
Cuando los dos electrones tienen igual projección de espín
$\chi_n(\omega)=\chi_m(\omega)=\alpha(\omega)$ o 
$\chi_n(\omega)=\chi_m(\omega)=\beta(\omega)$.
Entonces,
\begin{align}
 K_{n,m}
 &=\int 
 \frac{\phi_n^*(\mathbf{r}_1)\phi_m^*(\mathbf{r}_2)\phi_m(\mathbf{r}_1)\phi_n(\mathbf{r}_2)}{r_{12}} 
 \,d\mathbf{r}_1\,d\mathbf{r}_2 
 \underbrace{\int \alpha^*(\omega_1)\alpha(\omega_1)\,d\omega_1}_{=1}
 \underbrace{\int \alpha^*(\omega_2)\alpha(\omega_2) \,d\omega_2}_{=1}\\
 &=\int 
 \frac{\phi_n^*(\mathbf{r}_1)\phi_m^*(\mathbf{r}_2)\phi_m(\mathbf{r}_1)\phi_n(\mathbf{r}_2)}{r_{12}} 
 \,d\mathbf{r}_1\,d\mathbf{r}_2 
\end{align}
Cuando los electrones tienen projección de espín opuesta,
$\chi_n(\omega)=\alpha(\omega)$ y $\chi_m(\omega)=\beta(\omega)$ o 
$\chi_n(\omega)=\beta(\omega)$ y $\chi_m(\omega)=\alpha(\omega)$.
Entonces,
\begin{align}
 K_{n,m}
 &=\int 
 \frac{\phi_n^*(\mathbf{r}_1)\phi_m^*(\mathbf{r}_2)\phi_m(\mathbf{r}_1)\phi_n(\mathbf{r}_2)}{r_{12}} 
 \,d\mathbf{r}_1\,d\mathbf{r}_2 
 \underbrace{\int \alpha^*(\omega_1)\beta(\omega_1)\,d\omega_1}_{=0}
 \underbrace{\int \alpha^*(\omega_2)\beta(\omega_2) \,d\omega_2}_{=0}\\
 &=0
\end{align}



\end{document}
