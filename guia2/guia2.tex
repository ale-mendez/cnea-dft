\documentclass[10pt]{article}
%\usepackage{wallpaper}
\usepackage[spanish]{babel}
\usepackage[utf8]{inputenc}
\usepackage[top=1in, bottom=1.0in, left=1.in, right=1.in]{geometry}

\usepackage{amsmath}
\usepackage{graphicx}
\usepackage{animate}
\usepackage{amssymb}
\usepackage{gensymb}
\usepackage[makeroom]{cancel}
\usepackage{mathtools}% Loads amsmath
\usepackage{tikz}

\newcommand{\angstrom}{\text{\normalfont\AA}}
\newlength\myheight
\newcommand*\ccircled[1]{\settowidth{\myheight}{#1}%
    \raisebox{-.1\myheight}{\tikz[baseline=(char.base)]{%
        \node[shape=circle,draw,minimum size=1.5em,inner sep=1pt](char){#1};}}}
        
        
\pagestyle{empty}
\setlength\parindent{0pt}

\begin{document}

%\vspace{1.5cm}
%\hfill{Buenos Aires, \today} \\

\begin{center}
 {\large \bf Estructura Electrónica de Materias: \\
 Cálculo desde primeros principios} \\
 
 \vspace{0.25cm}
 Guía Práctica N\degree 2
\end{center}

\vspace{0.5cm}
1.  (a) Las posiciones de los átomos de la base del silicio en 
estructura diamante son: 
\begin{equation}
 \mathbf{B}_1=(0,0,0)\,\quad y\,\quad \mathbf{B}_2=\frac{a}{4}(1,1,1)\,,
\end{equation}
donde $a=5.43\angstrom$. Los vectores de la red de Bravais están 
dados por:
\begin{align}
 \mathbf{a}_1=\frac{a}{2} \left(0,1,1\right)\angstrom\,,\qquad
 \mathbf{a}_2=\frac{a}{2} \left(1,0,1\right)\angstrom\,,\qquad
 \mathbf{a}_3=\frac{a}{2} \left(1,1,0\right)\angstrom\,.
\end{align}

(b) Las posiciones de los átomos de la base del silicio en 
estructura hcp son: 
\begin{equation}
 \mathbf{B}_1=(0,0,0)\,\quad y\,\quad 
 \mathbf{B}_2=a\left(\tfrac{1}{3},\tfrac{2}{3},\tfrac{1}{2}\right)\angstrom\,,
\end{equation}
donde $a=2.5424\angstrom$
Los vectores de la red de Bravais están dados por:
\begin{align}
 \mathbf{a}_1 = a \left(1,0,0\right)\,,\qquad
 \mathbf{a}_2 = a \left(-\tfrac{1}{2},\tfrac{\sqrt{3}}{2},0\right)\,,\qquad
 \mathbf{a}_3 = c \left(0,0,1\right)\,,
\end{align}
y $c=4.1695\angstrom$.

\vspace{0.5cm}
2. La estructura cristalina del SrTiO$_3$ es simple cúbica y se puede 
describir como una
red fcc con iones de Sr$^{2+}$ y O$^{2-}$ con iones de Ti$^{4+}$ en 
los huecos octaédricos creados por los iones de oxigeno. En la unidad
cúbica, los iones de Sr, Ti y O se ubican, respectivamente, en:
\begin{equation}
 \mathbf{B}_1= \left(0,0,0\right)\angstrom\,\quad
 \mathbf{B}_2= \frac{a}{2}\left(1,1,1\right)\angstrom\,\quad
 \mathbf{B}_3= \frac{a}{2}\left(1,1,0\right)\angstrom\,\quad
\end{equation}
donde $a=3.905\angstrom$.

\vspace{0.5cm}
3. (a) Archivo POSCAR de la estructura diamante del Si:
\begin{verbatim}
Si - Diamond 
5.43
0.00  0.50  0.50
0.50  0.00  0.50
0.50  0.50  0.00
2
Cartesian
0.00  0.00  0.00
0.25  0.25  0.25
\end{verbatim}
(b) Archivo POSCAR de la estructura hcp del Si:
\begin{verbatim}
Si - hcp
2.5424
 1.000000000  0.000000000  0.000000000
-0.500000000  0.866025404  0.000000000
 0.000000000  0.000000000  1.640000000
2
Cartesian
 0.000000000  0.000000000  0.000000000
 0.333333333  0.666666667  0.500000000
\end{verbatim}
(c) Archivo POSCAR de la estructura del SrTiO$_3$:
\begin{verbatim}
perovskita SrTiO3 
3.905
1.0  0.0  0.0
0.0  1.0  0.0 
0.0  0.0  1.0
1 1 3
Cartesian
0.0  0.0  0.0
0.5  0.5  0.5
0.0  0.5  0.5
0.5  0.0  0.5
0.5  0.5  0.0
\end{verbatim}

4. Script en bash para variar el volumen en el archivo POSCAR:
\begin{verbatim}
ini=4.4
fin=6.6
step=0.2

for a in `seq $ini $step $fin`
do

mkdir vol_$a
cp KPOINTS POTCAR INCAR vol_$a/
cat > POSCAR <<!
diamond Si
$a
0.0 0.5 0.5
0.5 0.0 0.5
0.5 0.5 0.0
2
Cartesian
0.0  0.0  0.0
0.25 0.25 0.25
!
\end{verbatim}
donde {\verb ini } y {\verb fin } son los valores mínimo y máximo
a evaluar del parámetro de red $a$ y {\verb step } es el espaciado 
entre los valores a evaluar.

\end{document}
